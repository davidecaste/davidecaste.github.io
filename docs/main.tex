%%%%%%%%%%%%%%%%%%%%%%%%%%%%%%%%%%%%%%%%%
% "ModernCV" CV and Cover Letter
% LaTeX Template
% Version 1.11 (19/6/14)
%
% This template has been downloaded from:
% http://www.LaTeXTemplates.com
%
% Original author:
% Xavier Danaux (xdanaux@gmail.com)
%
% License:
% CC BY-NC-SA 3.0 (http://creativecommons.org/licenses/by-nc-sa/3.0/)
%
% Important note:
% This template requires the moderncv.cls and .sty files to be in the same 
% directory as this .tex file. These files provide the resume style and themes 
% used for structuring the document.
%
%%%%%%%%%%%%%%%%%%%%%%%%%%%%%%%%%%%%%%%%%

%----------------------------------------------------------------------------------------
%	PACKAGES AND OTHER DOCUMENT CONFIGURATIONS
%----------------------------------------------------------------------------------------


\documentclass[11pt,a4paper,sans]{moderncv} % Font sizes: 10, 11, or 12; paper sizes: a4paper, letterpaper, a5paper, legalpaper, executivepaper or landscape; font families: sans or roman

%\usepackage{geometry}
%\geometry{top=2cm}
\usepackage{comment}
\moderncvstyle{classic} % CV theme - options include: 'casual' (default), 'classic', 'oldstyle' and 'banking'
\moderncvcolor{blue} % CV color - options include: 'blue' (default), 'orange', 'green', 'red', 'purple', 'grey' and 'black'

\usepackage{lipsum} % Used for inserting dummy 'Lorem ipsum' text into the template
\newcommand{\cvreference}[7]{%
    \textbf{#1}\newline% Name
    \ifthenelse{\equal{#2}{}}{}{\addresssymbol~#2\newline}%
    \ifthenelse{\equal{#3}{}}{}{#3\newline}%
    \ifthenelse{\equal{#4}{}}{}{#4\newline}%
    \ifthenelse{\equal{#5}{}}{}{#5\newline}%
    \ifthenelse{\equal{#6}{}}{}{\emailsymbol~\texttt{#6}\newline}%
    \ifthenelse{\equal{#7}{}}{}{\phonesymbol~#7}}
\usepackage[scale=0.8]{geometry} % Reduce document margins
\setlength{\hintscolumnwidth}{4cm} % Uncomment to change the width of the dates column
\setlength{\makecvtitlenamewidth}{10cm} % For the 'classic' style, uncomment to adjust the width of the space allocated to your name


%----------------------------------------------------------------------------------------
%	NAME AND CONTACT INFORMATION SECTION
%----------------------------------------------------------------------------------------

\firstname{Davide} % Your first name
\familyname{Castelnovo} % Your last name

% All information in this block is optional, comment out any lines you don't need
\title{Ph. D.}
\address{Civate (LC), Italy, 23862}
\mobile{(+39) 3348908828}
\email{castelnovod@gmail.com}
\homepage{https://davidecaste.github.io}  % The first argument is the url for the clickable link, the second argument is the url displayed in the template - this allows special characters to be displayed such as the tilde in this example
%\homepage{}{Homepage}
\extrainfo{Citizenship: Italian}
%\photo[70pt][0.4pt]{pictures/House} % The first bracket is the picture height, the second is the thickness of the frame around the picture (0pt for no frame)
%\quote{"A witty and playful quotation" - John Smith}
%----------------------------------------------------------------------------------------

\begin{document}

\makecvtitle % Print the CV title
%----------------------------------------------------------------------------------------
%EDUCATION SECTION
%----------------------------------------------------------------------------------------

\section{Education}

\cventry{Sep. 2024 - currently}{Research Fellow}{}{University of Pisa, Italy}{}{}

\cventry{Jun. 2023 - Jun. 2024}{Research Fellow}{}{University of Padova, Italy}{}{}

\cventry{Nov. 2019 - Oct. 2023}{Ph. D.}{}{University of Udine, Italy}{}
{\begin{itemize}
  \item Thesis: \textit{``Fuzzy algebraic theories and $\mathcal{M}, \mathcal{N}$-adhesive categories"}
  \item Supervisor: Prof. Marino Miculan
\end{itemize}
}

\cventry{Jan. - May 2022}{Visiting student at Tallinn Institute of Technology}{}{Estonia}{Collaborating with Dr. Fosco Loregian}
{}

\cventry{Oct. 2016 - Dec. 2018}{Master in Mathematics}{}{University of Milano,Italy}{}
{\begin{itemize}
  %\item Majors: General Relativity, Cosmology, The Physical Universe.
  \item Master thesis \textit{``A conservative embedding of intuitionistic logic
  	in a topos of sheaves"}
  \begin{itemize}
      \item Supervisors: Prof. S. Mantovani (Univ. of Milano) \& Prof. M. E. Maietti (Univ. of Padova).
  \end{itemize}
  \item GPA: 110/110 summa cum laude.
\end{itemize}
}


\cventry{Oct. 2012 - Jul. 2016}{Bachelor in Mathematics}{}{University of Milano, Italy}{}
{\begin{itemize}
  \item GPA: 99/110
\end{itemize}
}

\cventry{Sep. 2007 - Jun. 2012}{High School}{}{Liceo Scientifico Tecnologico ``Antonio Badoni", Lecco, Italy}{}
{\begin{itemize}
  \item GPA: 97/100
\end{itemize}
}

\iffalse 
\section{Awards}

\cventry{2019}{Ph.D. Scholarship}{SFB 956 collaboration}{Project C4: Star formation and feedback in simulations of galaxy formation}{AIfA}{}{}

\cventry{2019}{Ph.D. Scholarship}{Tor Vergata University}{Rome}{(declined)}{}{}

\cventry{March 2019 - June 2019}{Undergraduate fellowship}{Laboratory-of-Excellence OCEVU undergraduate fellowship}{Under the supervision of Dr. Carlo Schimd}{Laboratoire d'Astrophysique de Marseille}{}

\cventry{Sept. 2018 - June 2019}{Erasmus at Aix-Marseille Universit\'e}{Master 2 - Fundamental Physics}{France}{}
{\begin{itemize}
  \item Majors: The Relativistic Universe, Galaxies and Cosmology, Astroparticles.
  \item Project on \textit{"CMB fluctuations: from Planck maps to cosmological parameters"}. Supervisor: Prof. G. Lagache.
  %\item Stage at Observatoire de Haute-Provence: observations with the 0.80m and 1.20m telescopes (interferometry of Solar System objects; imaging of galactic and extra-galactic objects).
  %\item Scientific Watch on \textit{"Collider constraints on light mediators connecting the SM to the dark sector"}. Supervisor: Dr. A. Bharucha.
  \item Weighted Average Mark: 16.60/20.00 (best 5\% of the second year Physics Master students).
\end{itemize}
}

%----------------------------------------------------------------------------------------
%	Collaborations SECTION
%----------------------------------------------------------------------------------------

\section{International collaboration and membership}

\cventry{Gen. 2020 - today}{Member of the International Max Planck Research School for Astronomy $\&$ Astrophysics (IMPRS)}{Max Planck Institute for Radio Astronomy}{Germany}{}{}

\cventry{Nov. 2019 - today}{Member of the SFB 956 collaboration}{project C4 (P.I. Prof. Dr. Cristiano Porciani)}{Germany}{}{}

\section{Certifications}
\cventry{2021}{Solution focus and resilience for successful careers}{SFB 956}{(8hrs, online)}{}{}
\cventry{2020}{Didactics Seminar: Teaching Teaching, Understanding Understanding}{IMPRS}{(16hrs, online)}{}{}
\fi 



%----------------------------------------------------------------------------------------
%	Skills SECTION
%----------------------------------------------------------------------------------------

\section{Technical skills and interests}

\hspace{15pt}My main research activity lies at the intersection of \textbf{Category Theory} and theoretical computer science. In particular, I have worked in the field of \textbf{Algebraic Graph Transformation}, where I have produced extensive theoretical work on \textbf{Double Pushout (DPO) Rewriting Systems}, an abstract mathematical framework for graph rewriting based on two steps approach: first, deleting the part of the graph required to apply a given rule, and then inserting a new graph into the resulting ``hole''.


\hspace{15pt}My academic work has given me a highly abstract and theoretical perspective in computer science and formal methods. To complement this, I have independently been improving my programming skills, particularly in \textbf{Haskell}, \textbf{Python}, and \textbf{Bash}. I also have extensive experties with the \LaTeX{} editor.

\section{Other interests}

\hspace{15pt}Outside of academia, my main passion is the mountains. Practically, this is reflected in my love for mountaineering and climbing, activities I have been practicing for over ten years. Intellectually, this passion extends to economic history, with a focus for the economic and demographic development of the  Alpine Arc and the related evolution in social institutions.
\section{Languages}

\cvitemwithcomment{Italian}{Mother tongue}{}
\cvitemwithcomment{English}{Fluent, speaking and writing}{}

\newpage 
\section{Talks and workshops}
\cventry{Sep. 2024}{Talk at 35$^{\mathbf{th}}$ International Conference on Concurrency Theory,  ``Left-Linear Rewriting in Adhesive Categories"}{University of Calgary}{Calgary}{Canada}{}
\cventry{Mar. 2024}{ Talk at Workshop on Process Theory for Security Protocols and Cryptography ``On the axioms of $\mathcal{M,N}$-adhesive categories".}{Tallinn Institute of Technology}{Tallinn}{Estonia}{}
\cventry{Feb. 2024}{  Talk at 35$^\mathbf{th}$ Estonian and the 9$^\mathbf{th}$ Joint Estonian-Latvian Theory Days  ``On the axioms of $\mathcal{M,N}$-adhesive categories".}{}{Randiv{\"a}lja}{Estonia}
{}
\cventry{Feb. 2024}{ Talk at TallCat - Compositional Systems and Methods group at TalTech, ``On the axioms of $\mathcal{M,N}$-adhesive categories".}{Tallinn Institute of Technology}{Tallinn}{Estonia}{}
\cventry{Jul. 2023}{  Talk at PRIN 2017 ``IT-MATTERS" Final Workshop  ``$\mathcal{M,N}$-adhesivity: results and examples"}{Scuola IMT Alti Studi}{Lucca}{Italy}
{}
\cventry{Dec. 2022}{3$^{\mathbf{rd}}$ ItaCa Workshop}{University of Pisa}{Pisa, Italy}{}
{}


\cventry{Sep. 2022}{9$^{\mathbf{th}}$ Symposium on Compositional Structures}{University of Como}{Como, Italy}{}
{}

\cventry{Apr. 2022}{  Talk at 25$^{\mathbf{th}}$ International Conference on Foundations of Software Science and Computation Structures,  ``A new criterion for $\mathcal{M,N}$-adhesivity, with an application to hierarchical graphs"}{Technische Universit$\ddot{\textsf{a}}$t M$\ddot{\textsf{u}}$nchen}{M$\ddot{\textsf{u}}$nchen}{Germany}
{}

\cventry{Feb. 2022}{Talk at 30$^{\mathsf{th}}$ EACSL Annual Conference on Computer Science in Logic 2022, ``Fuzzy Algebraic theories"}{University of G$\ddot{\textsf{o}}$ttingen}{G$\ddot{\textsf{o}}$ttingen (online)}{Germany}{}{}
{}


\cventry{Mar. 2021}{Talk at {22}$^{\mathbf{nd}}$ Graduate Student Conference in Logic ``Closure Hyperdoctrines''}{University of Illinois}{Urbana-Champaign (online)}{United States of America}{}


\cventry{Sep. 2021}{International Conference on Algebra, Topology and Their Interactions}{Udine (online)}{Italy}{}{}

\cventry{Sep. 2021}{Talk at {9}$^{\mathbf{th}}$ Conference  Conference on Algebra and Coalgebra in Computer Science, ``Closure Hyperdoctrines''}{University of Salzburg}{Salzburg}{Austria}{}

\cventry{Apr. 2021}{Midlands Graduate School in the Foundations of Computing Science}{University of Sheffield}{Sheffield (online)}{United Kingdom}
{}


\cventry{Oct. 2020}{ Talk at TallCat - Compositional Systems and Methods group at TalTech, ``Equational theories in an enriched context".}{Tallinn Institute of Technology}{Tallinn (online)}{Estonia}
{}

\cventry{Mar. 2020}{Talk at {25}$^{\mathbf{th}}$ Estonian Winter School in Computer Science, ``Preclosure Hyperdoctrines, Completeness of (Higher Order) Spatial Logic for Closure Spaces''}{Tallinn Institute of Technology}{Palmse}{Estonia}{}

\cventry{Dec. 2019}{1$^{\mathbf{st}}$ ItaCa Workshop}{University of Milano}{Milano, Italy}{}
{}

\newpage 
\section{Teaching and outreach activities}

\cventry{2020, 2021 (winter term)}{Undergraduate students tutoring}{}{Tutoring the \textit{Algebra 1} class for the first year of the Math undergraduate program, University of Udine}{}
{}

\cventry{2021, 2023 (winter term)}{Undergraduate students tutoring}{}{Tutoring the \textit{Linear Algebra} for the first year of the Mechanical Engineering undergraduate program, University of Udine}{}
{}

\section{Other Scientific Activity}
\cventry{Sep. 7-8, 2021}{Member of the Organizing Committee of the International Conference on Algebra, Topology and Their Interactions}{Udine (online)}{Italy}{}
{}

\cventry{2020 - 2023}{Students' representative at Mathematics, Computer Science and Physics Departmental Council}{Univ. of Udine}{Italy}{}
{}
\cventry{2021 - 2023}{Students' representative at Mathematics, Computer Science and Physics PhD School Council}{University of Udine}{Italy}{}
{}


%\section{Technical Skills}
%
%\cvitem{Programming}{\textsc{Haskell} (basic), \textsc{C++} (basic)}
%\cvitem{OS}{\textsc{Linux, Windows}}
%\cvitem{Editing}{\textsc{\LaTeX, Office Package}}

%----------------------------------------------------------------------------------------
%	LANGUAGES SECTION
%----------------------------------------------------------------------------------------

\iffalse 
\section{References}

\cventry{Prof.}{Marino Miculan}{}{\newline \hspace{0.5cm} email: marino.micula@uniud,it}{\newline \hspace{0.5cm} Argelander Institut f$\mathrm{\ddot{u}}$r Astronomie der Universit$\ddot{a}$t Bonn}{}{}

\cventry{Prof. Dr.}{Sabino Matarrese}{}{\newline \hspace{0.5cm} email: sabino.matarrese@pd.infn.it}{\newline \hspace{0.5cm} Department of Physics and Astronomy - Galileo Galilei, University of Padova}{INFN, Padova Section; INAF - Astronomical Observatory of Padova}{ \hspace{2.4cm} Gran Sasso Science Institute}{\newline}{}

\cventry{Dr.}{Carlo Schimd}{}{\newline \hspace{0.5cm} email: carlo.schimd@lam.fr}{\newline \hspace{0.5cm} Laboratoire d'Astrophysique de Marseille}{}{}

\cventry{Dr.}{Enrico Garaldi}{}{\newline \hspace{0.5cm} email: egaraldi@MPA-Garching.MPG.DE}{\newline \hspace{0.5cm} Max-Planck-Institut fuer Astrophysik}{}{}
\fi 

\section{List of publications}

\subsection{Journals}
\cvitem{2025}{Davide Castelnovo, Marino  Miculan, \emph{On The Axioms Of $\mathcal {M},\mathcal {N} $-Adhesive Categories}, Logical Methods in Computer Science, 21.}{}
\cvitem{2024}{Davide Castelnovo, Fabio Gadducci, Marino Miculan, \emph{A simple criterion for $\mathcal{M, N}$-adhesivity}, Theoretical Computer Science 982:114280}{}
\subsection{Proceedings of conferences}
\cvitem{2024}{Paolo Baldan, Davide Castelnovo, Andrea Corradini, Fabio Gadducci, \emph{Left-Linear Rewriting in Adhesive Categories}, Proceedings of the 35$^{\mathbf{th}}$ International Conference on Concurrency Theory}
\cvitem{2022}{Davide Castelnovo, Fabio Gadducci, Marino Miculan, \emph{A new criterion for $\mathcal{M, N}$-adhesivity, with an application to hierarchical graphs}, Proceedings of the 25$^{\mathbf{th}}$ International Conference on Foundations of Software Science and Computation Structures}{}
\cvitem{2022}{Davide Castelnovo, Marino Miculan, \emph{Fuzzy Algebraic Theories},
	Proceedings of the 30$^\mathsf{th}$ EACSL Annual Conference on Computer Science Logic}{}
\cvitem{2021}{Davide Castelnovo, Marino Miculan, \emph{Closure Hyperdoctrines},
	Proceedings of the 9$^{\mathsf{th}}$ Conference on Algebra and Coalgebra in Computer Science}{}



%----------------------------------------------------------------------------------------
%	COVER LETTER
%----------------------------------------------------------------------------------------

% To remove the cover letter, comment out this entire block

%\clearpage

%\recipient{HR Department}{Corporation\\123 Pleasant Lane\\12345 City, State} % Letter recipient
%\date{\today} % Letter date
%\opening{Dear Sir or Madam,} % Opening greeting
%\closing{Sincerely yours,} % Closing phrase
%\enclosure[Attached]{curriculum vit\ae{}} % List of enclosed documents

%\makelettertitle % Print letter title

%\lipsum[1-3] % Dummy text

%\makeletterclosing % Print letter signature

%----------------------------------------------------------------------------------------

\end{document}



